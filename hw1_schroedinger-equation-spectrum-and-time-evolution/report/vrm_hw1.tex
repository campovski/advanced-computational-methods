\documentclass[11pt, a4paper]{article}
\usepackage[slovene]{babel}
\usepackage[utf8]{inputenc}
\usepackage[T1]{fontenc}
\usepackage{lmodern}
\usepackage{amsmath}
\usepackage{amsfonts}
\usepackage{graphicx}
\usepackage[left=2cm, right=2cm, top=2cm, bottom=2cm]{geometry}

\title{Visje racunske metode -- Naloga 1}
\author{Marcel Čampa}
\date{\today}

\begin{document}
    \maketitle

    \section{Metode}
    Uporabili smo tri metode za numericno resevanje parcialnih diferencialnih enacb.

    \subsection{Metoda koncnih diferenc}
    Najosnovnejsa metoda, pri kateri odvode aproksimiramo z izrazitvijo iz Taylorjeve vrste.
    Pri diskretizaciji naravno vzamemo, da sta clena $i$ in $i+1$ za $h$ narazen, kjer je $h$ iz
    Taylorjevega razvoja, ki si ga lahko izberemo sami. Tako dobimo formulo za casovni korak
    \begin{equation} \label{eq:finite_diff}
        \psi_{m,n+1} = \psi_{m,n} + i \tau \left\{ \frac{1}{2h^2} (\psi_{m+1,n} + \psi_{m-1,n}) - 2 \psi_{m,n}) - V_m \psi_{m,n} \right\}.
    \end{equation}

    \subsection{Skoki s koncnim propagatorjem}
    Ce razvijemo propagator za koncni korak $\tau$ v vrsto, dobimo
    \begin{equation}
        \psi_{m,n+1} = \left\{ \exp(-i\tau \bar{H}) \psi \right\}_{m,n} = \sum_{k=0}^K \frac{(-i\tau)^k}{k!} (\bar{H}^k \psi)_{m,n},
    \end{equation}
    kjer si $K$ izberemo poljubno. Seveda velja, da vecji kot je $K$, manjsa je napaka, ki je reda $\mathcal{O}(\tau^{K+1})$. Mi smo si
    izbrali $K = 10$.

    \subsection{Implicitna shema}
    Tokrat propagator aproksimiramo z
    $$\exp(-i\tau \bar{H}) = \left( 1 + i \frac{\tau}{2} \bar{H} \right)^{-1} \left( 1 - i\frac{\tau}{2} \bar{H} \right).$$
    Napaka aproksimacije je reda $\mathcal{O}(\tau^3)$. Ta aproksimacija je se posebej dobra zato, ker je tudi unitaren operator,
    zaradi cesa bo $\psi$ se vedno ostala pravilne oblike. Ce sedaj zapisemo casovni razvoj s koncnimi diferencami,
    dobimo implicitno enacbo
    \begin{multline*}
        \psi_{m,n+1} - \frac{i\tau}{4} \left\{ \frac{1}{h^2} (\psi_{m+1,n+1} + \psi_{m-1,n+1} - 2 \psi_{m,n+1}) - 2V_m \psi_{m,n+1} \right\} = \\
        = \psi_{m,n} + \frac{i\tau}{4} \left\{ \frac{1}{h^2} (\psi_{m+1,n} + \psi_{m-1,n} - 2 \psi_{m,n}) - 2 V_m \psi_{m,n} \right\},
    \end{multline*}
    kar lahko zapisemo s tridiagonalno matricno enacbo
    \begin{equation}
        \begin{bmatrix}
            1 + \frac{i\tau}{2h^2} + \frac{i\tau}{2}V_1 & - \frac{i\tau}{4h^2} & & &\\
            - \frac{i\tau}{4h^2} & 1 + \frac{i\tau}{2h^2} + \frac{i\tau}{2}V_2 & - \frac{i\tau}{4h^2} & &\\
            & \ddots & \ddots & \ddots & &\\
            & & - \frac{i\tau}{4h^2} & 1 + \frac{i\tau}{2h^2} + \frac{i\tau}{2}V_{m-1} &- \frac{i\tau}{4h^2}\\
            & & &  - \frac{i\tau}{4h^2} & 1 + \frac{i\tau}{2h^2} + \frac{i\tau}{2}V_m
        \end{bmatrix} \begin{bmatrix}
            \psi_{1,n+1} \\ \psi_{2,n+1} \\ \vdots \\ \psi_{m,n+1}
        \end{bmatrix} = \text{RHS},
    \end{equation}
    kjer je RHS enako desni strani v enacbi \ref{eq:finite_diff}.


    \section{Naloge}

    \subsection*{Naloga 1}
    Obravnavali smo anharmonski oscilator $$V(x) = \frac{1}{2} x^2 + \lambda x^4.$$
    Za zacetno valovno funkcijo smo vzeli razlicne lastne funkcije harmonskega oscilatorja
    $$\phi_N(x) = \frac{1}{\pi^{1/4} \sqrt{2^N N!}} \, H_N(x) \exp(-x^2/2),$$
    kjer je $H_N(x)$ $N$-ti Hermitov polinom, ki ga v Pythonu izracunamo s pomocjo ze implementirane funkcije
    za Hermitovo vrsto \texttt{numpy.polynomial.hermite.hermval(x, c)}.

    Implementirali smo vse tri zgoraj opisane metode, vendar dobro deluje le zadnja, ki je
    unitarna, pri vseh ostalih prej ali slej stvar zdivergira v neskoncno. Za $\lambda = 0$ dobimo
    tako, da se v casu stvar sploh ne razvija, kar sledi iz premisleka, da je zacetna funkcija $\varphi_N(x)|_{\lambda=0}$ lastna
    funkcija operatorja $\hat{H}$, kar pomeni, da se ne spreminja s casom. Omenjeno stanje prikazuje slika
    \ref{fig:naloga1_lambda0}.

    \begin{figure}
        \includegraphics[width=\textwidth]{../images/final/naloga1_N_lambda_0.pdf} 
        \caption{Valovna funkcija je pri $\lambda = 0$ konstantno enaka zacetni funkciji
        $\phi_N(x)$.}
        \label{fig:naloga1_lambda0}
    \end{figure}

    Poleg tega smo naredili dve animaciji casovnega razvoja. Obe prikazujeta casovni razvoj pri $N = 0,1,\dotsc,5$ in
    $\lambda = 0, 0.25, 0.5$, pri eni na isti plot risemo razvoje z istim $N$ (\texttt{naloga1\_N.mp4}), pri drugi pa razvoje z isto $\lambda$
    (\texttt{naloga1\_lambda.mp4}).

    \subsection{Naloga 2}

    Ta naloga je bila zelo podobna prejsnji, le da smo sedaj premaknili stvar za $a > 0$ v desno, torej
    smo za zacetno funkcijo vzeli $\phi_0(x-a)$. Vzeli smo $a = 5$, saj smo ves cas delali na intervalu $[-10, 10]$,
    veljati pa mora $L - a \gg 1$. Nato smo pocasi spreminjali $\lambda$, in sicer smo vzeli $\lambda \in \{ 0, 0.05, 0.1, 0.15, 0.2, 0.25, 0.3, 0.35, 0.4, 0.45, 0.5\}$.
    Stvar je pricakovano postajala vedno bolj kaoticna, casovno evolucijo pa prikazuje animacija \texttt{naloga2.mp4}.

    Pri tej nalogi pa smo si dali malo duska. Vzeli smo zacetno stanje $\phi_0(x)$ in naredili dve odvisnosti:
    \begin{enumerate}
        \item Cas, da pride maksimum valovne funkcije v $x = 0$ v odvisnosti od $L$, pri cemer smo vzeli $a = L/2$.
        \item Cas, da pride maksimum valovne funkcije v $x = 0$ pri konstantnem $L=10,20$ v odvisnosti od $a$. Tu seveda zaradi pogoja $L-a \gg 1$ nismo racunali za $a$ ,,prevec'' blizu $L$.
    \end{enumerate}
    Dobili smo tocno take rezultate, ki smo jih pricakovali. Stvar je pri konstantnem $L$ neodvisna od izbire $a$ za $a > 0$, kar nam je jasno
    po preprostem premisleku, da vecji kot je $a$, vecji je potencial, torej posledicno vecja sila in hitrost potovanja vrha.
    To je lepo razvidno s slike \ref{fig:naloga2_a}.

    \begin{figure}
        \centering
        \includegraphics[width=0.8\textwidth]{../images/interesting/naloga2_iters_of_gaussian_travel_fixedL=10.pdf}
        \caption{Odvisnost potovanja vrha valovne funkcije do $x=0$ v odvisnosti od $a$ pri $L=10$. Opazimo, da se pri $a=7$,
        torej $L-a = 3$ zacne dogajati napaka zaradi nezanemarljive ploscine pod Gaussovko. $a = 0, 0.25, 0.50, 0.75, \dotsc, 8.50,8.75$.}
        \label{fig:naloga2_a}
    \end{figure}

    Ta in naslednji rezultat sta precej ocitna tudi z matematicnega vidika, saj je $$\int_{-\infty}^{-3} |\psi(x,t)|^2\, \text{d}x + \int_3^{\infty} |\psi(x,t)|^2\, \text{d}x < 0.003$$
    kar pomeni, da ce vzamemo $L-a=3$ ali pa $L-a=100$, ne naredimo skoraj nobene razlike. To je razvidno s slike \ref{fig:naloga2_L}.

    \begin{figure}
        \centering
        \includegraphics[width=0.8\textwidth]{../images/interesting/naloga2_iters_of_gaussian_travel.pdf}
        \caption{Odvisnost casa, da pride maksimum valovne funkcije do $x=0$ v odvisnosti od $L$ (in $a=L/2$). Rezultati so pricakovani,
        ce ne upostevamo nenatancnosti pri majhnih $L$.}
        \label{fig:naloga2_L}
    \end{figure}

    Ce pa pogledamo animaciji \texttt{naloga2\_L10\_a1.mp4} in \texttt{naloga2\_L10\_a5.mp4} pa vidimo, da se napaka zaradi
    zanemarjanja Gaussovke izven $[-L, L]$ in diskretizacije intervala pocasi nabira (pri $a=5$ v bistvu kar precej hitro). Za boljso natancnost oziroma
    daljsi cas do opazljive napake bi morali vzeti vecji $h$ in $L$, vendar bi na tako slabi masini, kot jo imam, izracun trajal
    celo vecnost, zato tega nisem storili.

    Naredili pa smo se graf razlike med tekoco valovno funkcijo in zacetno, kjer se lepo vidi, da je perioda neodvisna od $a$ za $L-a \gg 1$.
    Zasicenje okrog vrednosti $y=1$ pa se zgodi zato, ker je prekrivanje med ploscino pod tekoco valovno funkcijo in zacetno za dovolj velik $a$ vmes skoraj
    nicelno. Graf oddaljenosti smo normirali, tako vrednost $y=1$ pomeni, da valovni funkciji nimata skupne ploscine, ceprav je tam razlika ploscine dejansko $2$ ($1$ za vsako).
    Graf je prikazan na sliki \ref{fig:naloga2_lepa}.

    \begin{figure}
        \centering
        \includegraphics[width=0.8\textwidth]{../images/interesting/naloga2_distance.pdf}
        \caption{Razlika med tekoco valovno funkcijo in zacetno v odvisnosti od $a$ in iteracije. Zlahka vidimo, da je perioda neodvisna
        od $a$, kar smo ze dvakrat premislili v tekstu.}
        \label{fig:naloga2_lepa}
    \end{figure}
\end{document}